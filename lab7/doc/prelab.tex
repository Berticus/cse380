\documentclass[letterpaper,10pt]{article}

\setlength{\headheight}{0in}
\setlength{\marginparsep}{0in}
\setlength{\footskip}{0in}
\setlength{\headsep}{0in}
\setlength{\marginparwidth}{0in}
\setlength{\marginparpush}{0in}
\setlength{\voffset}{0in}
\setlength{\hoffset}{-1in}
\setlength{\voffset}{-1in}
\setlength{\oddsidemargin}{0.75in}
\setlength{\evensidemargin}{0.75in}
\setlength{\topmargin}{0.75in}
\setlength{\textheight}{9.5in}
\setlength{\textwidth}{7in}
\setlength{\parindent}{0in}
\setlength{\parskip}{10pt} %change this to match font size

\pagestyle{empty}

\begin{document}
    Albert Chang and Nipun Chopra\\
    CSE-380 A6\\
    University at Buffalo\\
    Dr. Kris Schindler\\
    March 22, 2011\\
    \textit{Lab 7 Prelab Write-up}

    The main screen is updated whenever there's movement, whether that be from
    the timer moving the barrels or Mario moving, coming from UART input. Both
    will have to be enabled to activate interrupts.

    When the timer interrupt is encountered, all barrels on screen are moved in
    the direction they need to move in. Barrels can only move down ladders, and
    once they reach the next level, the switch directions, The only time a
    barrel hits the wall is on the ground, and it disappears. When a barrel
    encounters any time of ladder, it may or may not fall down it.

    Three conditions need to be met before Mario can move. First, a key must be
    pressed to bring up the interrupt, and the key must be valid. Second, the
    direction needs to be valid. So on a ladder, vertical movements are enabled,
    when there's something under him, jumps and lateral movements are enabled.
    Third, the next move must be a valid spot. In the event there's nothing
    under him, movements are disabled and Mario falls. A fall of at least two
    characters causes Mario to lose a life. If Mario moves, the screen needs
    to be updated.

    The score changes whenever Mario jumps over a barrel or when the end of
    the level is reached. During these times, the score needs to be checked. At
    every 1000 points, the life is increased. 5 times the level points are
    rewarded for being over a barrel during a jump. 100 times level completed
    is the amount rewarded upon completing the level.

    If a barrel collides with Mario, Mario loses a life. The LEDs are
    updated as needed. Barrel collision is detected when Mario and a barrel
    occupy the same spot. This should be checked anytime the screen updates.
    The only other time the LEDs need to be updated is when Mario gains a life.

    The seven-segment display indicate what level we're on. Every time the level
    increases (reach princess), it updates. When Mario is in the position in
    front of the princess, the end of the level is reached.

    The third interrupt is caused by the push button, which pauses the game.
    "PAUSE" is printed at the center of the board. This disables any possible
    movements. As described before, that would be the timer and UART. Either
    the timer and UART interrupts can be disabled, or the timer and UART themselves
    can be disabled.

    Ejecting a barrel can be done either using a separate match counter register
    on the same timer (need to be careful when resetting timer) or using a separate
    timer altogether.

    Each barrel needs a word to store all relevant information. Mario needs a
    word to store his information. Life and level each require a nibble, they
    can be coupled into a byte.

    A random number generator is needed to decide whether a barrel falls down a
    ladder or continues. To seed the number generator, the real time clock can
    be used. The algorithm to generate a random number can either be used in a
    linear congruential sequence generator or through a linear feedback shift
    register. Testing will reveal which one is the better algorithm. The seed
    can also come from the digital analog converter.

    Another way random numbers can be generated is by reading from the digital
    analog converter, DAC. Any time a random number is needed, noise can be
    read from the device. Or similar to how \textit{/dev/random} and
    \textit{/dev/urandom}, in GNU/Linux, work, an entropy pool can be created
    by reading from multiple devices. This type of randomness is considered to
    be good in a cryptographically secure standpoint, but may not be optimal.
    Depending on which method is taken. By waiting for sufficient entropy in
    the entropy pool, time is wasted as well as RAM to hold all the numbers.
    However, by not waiting for sufficient entropy, entropy is not guaranteed.
    According to the documentation, these methods are strong, but slow, and
    is poorly suited for large amounts of random data.

    The tentative order of development is:

    \begin{enumerate}
        \item Make random number generator.
        \item Make barrels move at desired rate.
        \item Eject barrels at desired frequency.
        \item Make Mario move laterally.
        \item Make Mario move vertically.
        \item Make Mario jump.
        \item Implement pause.
        \item Process scores.
        \item Process lives.
        \item Process levels (adjust score, barrel speed, ejection frequency, display).
        \item Process collisions.
    \end{enumerate}

\end{document}
