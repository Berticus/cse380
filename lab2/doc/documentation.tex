\documentclass[letterpaper,10pt]{article}

\usepackage{listings}
\usepackage{color}
\usepackage{tikz}

\setlength{\headheight}{0in}
\setlength{\marginparsep}{0in}
\setlength{\footskip}{0in}
\setlength{\headsep}{0in}
\setlength{\marginparwidth}{0in}
\setlength{\marginparpush}{0in} \setlength{\voffset}{0in} \setlength{\hoffset}{-1in}
\setlength{\voffset}{-1in}
\setlength{\oddsidemargin}{0.75in}
\setlength{\evensidemargin}{0.75in}
\setlength{\topmargin}{0.75in}
\setlength{\textheight}{9.5in}
\setlength{\textwidth}{7in}
\setlength{\parindent}{0in}
\setlength{\parskip}{10pt} %change this to match font size

\pagestyle{empty}
\definecolor{gray}{gray}{0.75}
\usetikzlibrary{shapes,arrows,calc}
\lstset{numbers=left,backgroundcolor=\color{gray},frame=single}

% define block styles
\tikzstyle{line} = [draw, -latex']
\tikzstyle{block} = [draw, rectangle, text centered, minimum height=2em]
\tikzstyle{mlblock} = [draw, rectangle, text width=10em, text centered, minimum height=2em]
\tikzstyle{decision} = [draw, diamond, text width=4.5em, text centered, node distance=3cm, inner sep=0pt]
\tikzstyle{cloud} = [draw, rectangle, text centered, rounded corners, minimum height=2em]

\begin{document}
    Albert Chang and Nipun Chopra\\
    CSE-380 A6\\
    University at Buffalo\\
    Dr. Kris Schindler\\
    February 10, 2011\\
    \textit{Lab 2 Documentation}

    The objective of Lab 2 was to write a two routines called \textit{hamming}
    and \textit{div}. The \textit{hamming} routine receives an 8-bit hamming
    code and returns the corrected 4-bit data. Register r0 is used both to
    pass the 8-bit hamming code and to return the 4-bit data. The hamming
    code had been studied before in a previous class. With 8 bits, there would
    be 4 parity bits. The first checksum is calculated with by taking the
    sum of bits included by one bit, skipping a bit, taking another bit,
    skipping another, \textit{et cetera} until the end of the bitstream. The
    second checksum bit is calculated by taking two bits, skipping two.
    Likewise the third is takes four bits, skips four, although after it takes
    the four bits, there's nothing left in the bitstream. The checksum combined
    creates the address of the bit that needs to be flipped. Which means one
    error is known, and at that point corrected. However, at that point, if
    it makes the total parity odd, then there is another error, but it cannot
    be fixed. Contrarily, if fixing that bit makes the total parity even, then
    there are no more errors. The following flowchart displays how the
    algorithm was realized.

    Immediately, it was realized that there were multiple ways of doing each
    step of the process. For example, a \textit{bic} command could be used
    instead of \textit{and}. Or \textit{add} could be used instead of
    \textit{eor}. What ultimately decided which commands would be used came
    down to how fast each one worked---Keil displays the time it takes to
    execute the code---or flexibility and convenience. For example, by
    using an \textit{add} command, you would also need yet another command to
    extract the odd bit. With \textit{eor}, those lines could be eliminated.
    With \textit{bic} we were limited to the 8 bits we were working on---
    the rest would be 0s anyway---but if we ever modify the code, those
    \textit{bic} lines would have to be modified as well. The \textit{and}
    command accomplished the same thing, with the same execution time, but if
    we expanded to more than 8 bits, those lines wouldn't have to be modified.
    Using a revision control system, Git, it was easy to switch between
    different versions of code quickly without disrupting development of the
    main branch.

    \begin{center}
        \begin{tikzpicture}[node distance = 1.5cm, auto]
            \node[cloud] (origin) {Start};
            \node[decision, below of=origin] (large) {hamming $>$ 8-bit ?};
            \node[decision, below of=large] (small) {hamming $<$ 0 ?};
            \node[below of=small, node distance = 3.5cm] (empty1) {};
            \node[block, left of=empty1, node distance = 3cm] (eparity) {Count 1s in hamming code};
            \node[block, right of=empty1, node distance = 3cm] (isolate) {Isolate Data Bits};
            \node[block, below of=isolate] (parity1) {Isolate First Parity Bit};
            \node[block, below of=parity1] (checksum1) {Calculate First Checksum};
            \node[block, below of=checksum1] (parity2) {Isolate Second Parity Bit};
            \node[block, below of=parity2] (checksum2) {Calculate Second Checksum};
            \node[block, below of=checksum2] (parity3) {Isolate Third Parity Bit};
            \node[block, below of=parity3] (checksum3) {Calculate Third Parity Bit};
            \node[block, below of=checksum3] (address) {Create Error Address};
            \path[line] (origin) -- (large);
            \path[line] (large) -- node [near start] {no} (small);
            \path[line] (small) -- node [near start] {no} +(0,-2);
            \draw (eparity) |- +(6,1.5) -- (isolate);
            \path[line] (isolate) -- (parity1);
            \path[line] (parity1) -- (checksum1);
            \path[line] (checksum1) -- (parity2);
            \path[line] (parity2) -- (checksum2);
            \path[line] (checksum2) -- (parity3);
            \path[line] (parity3) -- (checksum3);
            \path[line] (checksum3) -- (address);
            \draw (address) |- +(-6,-1.5) -- (eparity);
            \draw (large) -| node [near start] {yes} +(7,-18.5);
            \draw (small) -- node [near start] {yes} +(7,0);
            %\path[line, dashed] (cont2) -- +(0,-1.375);
            \path[line, dashed] (address) +(4, -1.5) node {} -- +(4,-3);
            \path[line, dashed] (address) +(-3,-1.5) node {} -- +(-3,-3);
        \end{tikzpicture}
    \end{center}

    \begin{center}
        \begin{tikzpicture}[node distance = 1.5cm, auto]
            \node (cont) {};
            \node[left of=cont, node distance = 6.25cm] {};
            \node[decision, below of=cont] (errors) {Address $>$ 0 ?};
            \node[decision, below of=errors, node distance = 3cm] (correctable) {Parity of Total Is Even?};
            \node[block, below of=correctable, node distance = 3cm] (fix) {Correct Bit at Address};
            \node[block, right of=correctable, node distance = 7cm] (failure) {Register r0 = -1};
            \node[block, below of=fix] (extract) {Extract Data Bits};
            \node[cloud, below of=extract] (stop) {Stop};
            \path[line, dashed] (cont) -- (errors);
            \path[line, dashed] (cont) + (7,0) -- (failure);
            \path[line] (errors) -- node [near start] {yes} (correctable);
            \path[line] (correctable) -- node [near start] {yes} (fix);
            \path[line] (errors) --  node [near start] {no} +(-4,0) |- (extract);
            \path[line] (correctable) -- node [near start] {no} (failure);
            \path[line] (failure) |- (stop);
            \path[line] (fix) -- (extract);
            \path[line] (extract) -- (stop);
        \end{tikzpicture}
    \end{center}

    The \textit{div} routine returns the quotient of the value passed into
    r0 (dividend) divided by the value passed in r1 (divisor). Quotient is
    returned in r0. All values are unsigned integers. This relies on the flow
    chart provided during lecture. Some minor changes have been made to the
    flow to optimize speed, as is shown by the following flowchart.

    \begin{center}
        \begin{tikzpicture}[node distance = 1.5cm, auto]
            \node[cloud] (origin) {Start};
            \node[block, below of=origin] (initq) {Initialize Quotient to 0};
            \node[block, below of=initq] (initr) {Initialize Remainder};
            \node[block, below of=initr] (initc) {Initialize Counter to 16};
            \node[block, below of=initc] (lsldivor) {Logical Shift Left Divisor 16 Places};
            \node[block, below of=lsldivor] (subr) {Remainder := Remainder - Divisor};
            \node[decision, below of=subr] (rcheck) {Is\\Remainder $<$ 0 ?};
            \node[block, right of=rcheck, node distance = 7cm] (addr) {Remainder := Remainder + Divisor};
            \node[mlblock, below of=addr] (lsq0) {Logical Shift Quotient\\LSB = 0};
            \node[mlblock, below of=rcheck, node distance = 3cm] (lsq1) {Logical Shift Quotient\\LSB = 1};
            \node[mlblock, below of=lsq1] (msb) {Right Shift Divisor\\MSB=0};
            \node[decision, below of=msb] (ccheck) {Is Counter $\leq$ 1 ?};
            \node[block, left of=ccheck, node distance = 5cm] (dcount) {Decrement Counter};
            \node[cloud, below of=ccheck, node distance = 3cm] (stop) {Stop};
            \path[line] (origin) -- (initq);
            \path[line] (initq) -- (initr);
            \path[line] (initr) -- (initc);
            \path[line] (initc) -- (lsldivor);
            \path[line] (lsldivor) -- (subr);
            \path[line] (subr) -- (rcheck);
            \path[line] (rcheck) -- node [near start] {yes} (addr);
            \path[line] (addr) -- (lsq0);
            \path[line] (rcheck) -- node [near start] {no} (lsq1);
            \path[line] (lsq0) |- (msb);
            \path[line] (lsq1) -- (msb);
            \path[line] (msb) -- (ccheck);
            \path[line] (ccheck) -- node [near start] {yes} (dcount);
            \path[line] (dcount) |- (subr);
            \path[line] (ccheck) -- node [near start] {no} (stop);
        \end{tikzpicture}
    \end{center}
    % maybe insert pseudocode

    The day before Albert Chang and Nipun Chopra met, Albert had finished
    working on the actual code. Both the \textit{hamming} and \textit{div}
    routines were working. Since the coding aspect was done with, it only made
    sense that Nipun write up the documentation.

    hamming\_divide\_test\_wrapper.c:
    \lstinputlisting[language=C]{../hamming_divide_test_wrapper.c}

    hamming\_divide\_routines.s:
    \lstinputlisting{../hamming_divide_routines.s}
\end{document}
